
% Default to the notebook output style

    


% Inherit from the specified cell style.




    
\documentclass[11pt]{article}

    
    
    \usepackage[T1]{fontenc}
    % Nicer default font (+ math font) than Computer Modern for most use cases
    \usepackage{mathpazo}

    % Basic figure setup, for now with no caption control since it's done
    % automatically by Pandoc (which extracts ![](path) syntax from Markdown).
    \usepackage{graphicx}
    % We will generate all images so they have a width \maxwidth. This means
    % that they will get their normal width if they fit onto the page, but
    % are scaled down if they would overflow the margins.
    \makeatletter
    \def\maxwidth{\ifdim\Gin@nat@width>\linewidth\linewidth
    \else\Gin@nat@width\fi}
    \makeatother
    \let\Oldincludegraphics\includegraphics
    % Set max figure width to be 80% of text width, for now hardcoded.
    \renewcommand{\includegraphics}[1]{\Oldincludegraphics[width=.8\maxwidth]{#1}}
    % Ensure that by default, figures have no caption (until we provide a
    % proper Figure object with a Caption API and a way to capture that
    % in the conversion process - todo).
    \usepackage{caption}
    \DeclareCaptionLabelFormat{nolabel}{}
    \captionsetup{labelformat=nolabel}

    \usepackage{adjustbox} % Used to constrain images to a maximum size 
    \usepackage{xcolor} % Allow colors to be defined
    \usepackage{enumerate} % Needed for markdown enumerations to work
    \usepackage{geometry} % Used to adjust the document margins
    \usepackage{amsmath} % Equations
    \usepackage{amssymb} % Equations
    \usepackage{textcomp} % defines textquotesingle
    % Hack from http://tex.stackexchange.com/a/47451/13684:
    \AtBeginDocument{%
        \def\PYZsq{\textquotesingle}% Upright quotes in Pygmentized code
    }
    \usepackage{upquote} % Upright quotes for verbatim code
    \usepackage{eurosym} % defines \euro
    \usepackage[mathletters]{ucs} % Extended unicode (utf-8) support
    \usepackage[utf8x]{inputenc} % Allow utf-8 characters in the tex document
    \usepackage{fancyvrb} % verbatim replacement that allows latex
    \usepackage{grffile} % extends the file name processing of package graphics 
                         % to support a larger range 
    % The hyperref package gives us a pdf with properly built
    % internal navigation ('pdf bookmarks' for the table of contents,
    % internal cross-reference links, web links for URLs, etc.)
    \usepackage{hyperref}
    \usepackage{longtable} % longtable support required by pandoc >1.10
    \usepackage{booktabs}  % table support for pandoc > 1.12.2
    \usepackage[inline]{enumitem} % IRkernel/repr support (it uses the enumerate* environment)
    \usepackage[normalem]{ulem} % ulem is needed to support strikethroughs (\sout)
                                % normalem makes italics be italics, not underlines
    

    
    
    % Colors for the hyperref package
    \definecolor{urlcolor}{rgb}{0,.145,.698}
    \definecolor{linkcolor}{rgb}{.71,0.21,0.01}
    \definecolor{citecolor}{rgb}{.12,.54,.11}

    % ANSI colors
    \definecolor{ansi-black}{HTML}{3E424D}
    \definecolor{ansi-black-intense}{HTML}{282C36}
    \definecolor{ansi-red}{HTML}{E75C58}
    \definecolor{ansi-red-intense}{HTML}{B22B31}
    \definecolor{ansi-green}{HTML}{00A250}
    \definecolor{ansi-green-intense}{HTML}{007427}
    \definecolor{ansi-yellow}{HTML}{DDB62B}
    \definecolor{ansi-yellow-intense}{HTML}{B27D12}
    \definecolor{ansi-blue}{HTML}{208FFB}
    \definecolor{ansi-blue-intense}{HTML}{0065CA}
    \definecolor{ansi-magenta}{HTML}{D160C4}
    \definecolor{ansi-magenta-intense}{HTML}{A03196}
    \definecolor{ansi-cyan}{HTML}{60C6C8}
    \definecolor{ansi-cyan-intense}{HTML}{258F8F}
    \definecolor{ansi-white}{HTML}{C5C1B4}
    \definecolor{ansi-white-intense}{HTML}{A1A6B2}

    % commands and environments needed by pandoc snippets
    % extracted from the output of `pandoc -s`
    \providecommand{\tightlist}{%
      \setlength{\itemsep}{0pt}\setlength{\parskip}{0pt}}
    \DefineVerbatimEnvironment{Highlighting}{Verbatim}{commandchars=\\\{\}}
    % Add ',fontsize=\small' for more characters per line
    \newenvironment{Shaded}{}{}
    \newcommand{\KeywordTok}[1]{\textcolor[rgb]{0.00,0.44,0.13}{\textbf{{#1}}}}
    \newcommand{\DataTypeTok}[1]{\textcolor[rgb]{0.56,0.13,0.00}{{#1}}}
    \newcommand{\DecValTok}[1]{\textcolor[rgb]{0.25,0.63,0.44}{{#1}}}
    \newcommand{\BaseNTok}[1]{\textcolor[rgb]{0.25,0.63,0.44}{{#1}}}
    \newcommand{\FloatTok}[1]{\textcolor[rgb]{0.25,0.63,0.44}{{#1}}}
    \newcommand{\CharTok}[1]{\textcolor[rgb]{0.25,0.44,0.63}{{#1}}}
    \newcommand{\StringTok}[1]{\textcolor[rgb]{0.25,0.44,0.63}{{#1}}}
    \newcommand{\CommentTok}[1]{\textcolor[rgb]{0.38,0.63,0.69}{\textit{{#1}}}}
    \newcommand{\OtherTok}[1]{\textcolor[rgb]{0.00,0.44,0.13}{{#1}}}
    \newcommand{\AlertTok}[1]{\textcolor[rgb]{1.00,0.00,0.00}{\textbf{{#1}}}}
    \newcommand{\FunctionTok}[1]{\textcolor[rgb]{0.02,0.16,0.49}{{#1}}}
    \newcommand{\RegionMarkerTok}[1]{{#1}}
    \newcommand{\ErrorTok}[1]{\textcolor[rgb]{1.00,0.00,0.00}{\textbf{{#1}}}}
    \newcommand{\NormalTok}[1]{{#1}}
    
    % Additional commands for more recent versions of Pandoc
    \newcommand{\ConstantTok}[1]{\textcolor[rgb]{0.53,0.00,0.00}{{#1}}}
    \newcommand{\SpecialCharTok}[1]{\textcolor[rgb]{0.25,0.44,0.63}{{#1}}}
    \newcommand{\VerbatimStringTok}[1]{\textcolor[rgb]{0.25,0.44,0.63}{{#1}}}
    \newcommand{\SpecialStringTok}[1]{\textcolor[rgb]{0.73,0.40,0.53}{{#1}}}
    \newcommand{\ImportTok}[1]{{#1}}
    \newcommand{\DocumentationTok}[1]{\textcolor[rgb]{0.73,0.13,0.13}{\textit{{#1}}}}
    \newcommand{\AnnotationTok}[1]{\textcolor[rgb]{0.38,0.63,0.69}{\textbf{\textit{{#1}}}}}
    \newcommand{\CommentVarTok}[1]{\textcolor[rgb]{0.38,0.63,0.69}{\textbf{\textit{{#1}}}}}
    \newcommand{\VariableTok}[1]{\textcolor[rgb]{0.10,0.09,0.49}{{#1}}}
    \newcommand{\ControlFlowTok}[1]{\textcolor[rgb]{0.00,0.44,0.13}{\textbf{{#1}}}}
    \newcommand{\OperatorTok}[1]{\textcolor[rgb]{0.40,0.40,0.40}{{#1}}}
    \newcommand{\BuiltInTok}[1]{{#1}}
    \newcommand{\ExtensionTok}[1]{{#1}}
    \newcommand{\PreprocessorTok}[1]{\textcolor[rgb]{0.74,0.48,0.00}{{#1}}}
    \newcommand{\AttributeTok}[1]{\textcolor[rgb]{0.49,0.56,0.16}{{#1}}}
    \newcommand{\InformationTok}[1]{\textcolor[rgb]{0.38,0.63,0.69}{\textbf{\textit{{#1}}}}}
    \newcommand{\WarningTok}[1]{\textcolor[rgb]{0.38,0.63,0.69}{\textbf{\textit{{#1}}}}}
    
    
    % Define a nice break command that doesn't care if a line doesn't already
    % exist.
    \def\br{\hspace*{\fill} \\* }
    % Math Jax compatability definitions
    \def\gt{>}
    \def\lt{<}
    % Document parameters
    \title{Dr. Ignaz Semmelweis}
    
    
    

    % Pygments definitions
    
\makeatletter
\def\PY@reset{\let\PY@it=\relax \let\PY@bf=\relax%
    \let\PY@ul=\relax \let\PY@tc=\relax%
    \let\PY@bc=\relax \let\PY@ff=\relax}
\def\PY@tok#1{\csname PY@tok@#1\endcsname}
\def\PY@toks#1+{\ifx\relax#1\empty\else%
    \PY@tok{#1}\expandafter\PY@toks\fi}
\def\PY@do#1{\PY@bc{\PY@tc{\PY@ul{%
    \PY@it{\PY@bf{\PY@ff{#1}}}}}}}
\def\PY#1#2{\PY@reset\PY@toks#1+\relax+\PY@do{#2}}

\expandafter\def\csname PY@tok@w\endcsname{\def\PY@tc##1{\textcolor[rgb]{0.73,0.73,0.73}{##1}}}
\expandafter\def\csname PY@tok@c\endcsname{\let\PY@it=\textit\def\PY@tc##1{\textcolor[rgb]{0.25,0.50,0.50}{##1}}}
\expandafter\def\csname PY@tok@cp\endcsname{\def\PY@tc##1{\textcolor[rgb]{0.74,0.48,0.00}{##1}}}
\expandafter\def\csname PY@tok@k\endcsname{\let\PY@bf=\textbf\def\PY@tc##1{\textcolor[rgb]{0.00,0.50,0.00}{##1}}}
\expandafter\def\csname PY@tok@kp\endcsname{\def\PY@tc##1{\textcolor[rgb]{0.00,0.50,0.00}{##1}}}
\expandafter\def\csname PY@tok@kt\endcsname{\def\PY@tc##1{\textcolor[rgb]{0.69,0.00,0.25}{##1}}}
\expandafter\def\csname PY@tok@o\endcsname{\def\PY@tc##1{\textcolor[rgb]{0.40,0.40,0.40}{##1}}}
\expandafter\def\csname PY@tok@ow\endcsname{\let\PY@bf=\textbf\def\PY@tc##1{\textcolor[rgb]{0.67,0.13,1.00}{##1}}}
\expandafter\def\csname PY@tok@nb\endcsname{\def\PY@tc##1{\textcolor[rgb]{0.00,0.50,0.00}{##1}}}
\expandafter\def\csname PY@tok@nf\endcsname{\def\PY@tc##1{\textcolor[rgb]{0.00,0.00,1.00}{##1}}}
\expandafter\def\csname PY@tok@nc\endcsname{\let\PY@bf=\textbf\def\PY@tc##1{\textcolor[rgb]{0.00,0.00,1.00}{##1}}}
\expandafter\def\csname PY@tok@nn\endcsname{\let\PY@bf=\textbf\def\PY@tc##1{\textcolor[rgb]{0.00,0.00,1.00}{##1}}}
\expandafter\def\csname PY@tok@ne\endcsname{\let\PY@bf=\textbf\def\PY@tc##1{\textcolor[rgb]{0.82,0.25,0.23}{##1}}}
\expandafter\def\csname PY@tok@nv\endcsname{\def\PY@tc##1{\textcolor[rgb]{0.10,0.09,0.49}{##1}}}
\expandafter\def\csname PY@tok@no\endcsname{\def\PY@tc##1{\textcolor[rgb]{0.53,0.00,0.00}{##1}}}
\expandafter\def\csname PY@tok@nl\endcsname{\def\PY@tc##1{\textcolor[rgb]{0.63,0.63,0.00}{##1}}}
\expandafter\def\csname PY@tok@ni\endcsname{\let\PY@bf=\textbf\def\PY@tc##1{\textcolor[rgb]{0.60,0.60,0.60}{##1}}}
\expandafter\def\csname PY@tok@na\endcsname{\def\PY@tc##1{\textcolor[rgb]{0.49,0.56,0.16}{##1}}}
\expandafter\def\csname PY@tok@nt\endcsname{\let\PY@bf=\textbf\def\PY@tc##1{\textcolor[rgb]{0.00,0.50,0.00}{##1}}}
\expandafter\def\csname PY@tok@nd\endcsname{\def\PY@tc##1{\textcolor[rgb]{0.67,0.13,1.00}{##1}}}
\expandafter\def\csname PY@tok@s\endcsname{\def\PY@tc##1{\textcolor[rgb]{0.73,0.13,0.13}{##1}}}
\expandafter\def\csname PY@tok@sd\endcsname{\let\PY@it=\textit\def\PY@tc##1{\textcolor[rgb]{0.73,0.13,0.13}{##1}}}
\expandafter\def\csname PY@tok@si\endcsname{\let\PY@bf=\textbf\def\PY@tc##1{\textcolor[rgb]{0.73,0.40,0.53}{##1}}}
\expandafter\def\csname PY@tok@se\endcsname{\let\PY@bf=\textbf\def\PY@tc##1{\textcolor[rgb]{0.73,0.40,0.13}{##1}}}
\expandafter\def\csname PY@tok@sr\endcsname{\def\PY@tc##1{\textcolor[rgb]{0.73,0.40,0.53}{##1}}}
\expandafter\def\csname PY@tok@ss\endcsname{\def\PY@tc##1{\textcolor[rgb]{0.10,0.09,0.49}{##1}}}
\expandafter\def\csname PY@tok@sx\endcsname{\def\PY@tc##1{\textcolor[rgb]{0.00,0.50,0.00}{##1}}}
\expandafter\def\csname PY@tok@m\endcsname{\def\PY@tc##1{\textcolor[rgb]{0.40,0.40,0.40}{##1}}}
\expandafter\def\csname PY@tok@gh\endcsname{\let\PY@bf=\textbf\def\PY@tc##1{\textcolor[rgb]{0.00,0.00,0.50}{##1}}}
\expandafter\def\csname PY@tok@gu\endcsname{\let\PY@bf=\textbf\def\PY@tc##1{\textcolor[rgb]{0.50,0.00,0.50}{##1}}}
\expandafter\def\csname PY@tok@gd\endcsname{\def\PY@tc##1{\textcolor[rgb]{0.63,0.00,0.00}{##1}}}
\expandafter\def\csname PY@tok@gi\endcsname{\def\PY@tc##1{\textcolor[rgb]{0.00,0.63,0.00}{##1}}}
\expandafter\def\csname PY@tok@gr\endcsname{\def\PY@tc##1{\textcolor[rgb]{1.00,0.00,0.00}{##1}}}
\expandafter\def\csname PY@tok@ge\endcsname{\let\PY@it=\textit}
\expandafter\def\csname PY@tok@gs\endcsname{\let\PY@bf=\textbf}
\expandafter\def\csname PY@tok@gp\endcsname{\let\PY@bf=\textbf\def\PY@tc##1{\textcolor[rgb]{0.00,0.00,0.50}{##1}}}
\expandafter\def\csname PY@tok@go\endcsname{\def\PY@tc##1{\textcolor[rgb]{0.53,0.53,0.53}{##1}}}
\expandafter\def\csname PY@tok@gt\endcsname{\def\PY@tc##1{\textcolor[rgb]{0.00,0.27,0.87}{##1}}}
\expandafter\def\csname PY@tok@err\endcsname{\def\PY@bc##1{\setlength{\fboxsep}{0pt}\fcolorbox[rgb]{1.00,0.00,0.00}{1,1,1}{\strut ##1}}}
\expandafter\def\csname PY@tok@kc\endcsname{\let\PY@bf=\textbf\def\PY@tc##1{\textcolor[rgb]{0.00,0.50,0.00}{##1}}}
\expandafter\def\csname PY@tok@kd\endcsname{\let\PY@bf=\textbf\def\PY@tc##1{\textcolor[rgb]{0.00,0.50,0.00}{##1}}}
\expandafter\def\csname PY@tok@kn\endcsname{\let\PY@bf=\textbf\def\PY@tc##1{\textcolor[rgb]{0.00,0.50,0.00}{##1}}}
\expandafter\def\csname PY@tok@kr\endcsname{\let\PY@bf=\textbf\def\PY@tc##1{\textcolor[rgb]{0.00,0.50,0.00}{##1}}}
\expandafter\def\csname PY@tok@bp\endcsname{\def\PY@tc##1{\textcolor[rgb]{0.00,0.50,0.00}{##1}}}
\expandafter\def\csname PY@tok@fm\endcsname{\def\PY@tc##1{\textcolor[rgb]{0.00,0.00,1.00}{##1}}}
\expandafter\def\csname PY@tok@vc\endcsname{\def\PY@tc##1{\textcolor[rgb]{0.10,0.09,0.49}{##1}}}
\expandafter\def\csname PY@tok@vg\endcsname{\def\PY@tc##1{\textcolor[rgb]{0.10,0.09,0.49}{##1}}}
\expandafter\def\csname PY@tok@vi\endcsname{\def\PY@tc##1{\textcolor[rgb]{0.10,0.09,0.49}{##1}}}
\expandafter\def\csname PY@tok@vm\endcsname{\def\PY@tc##1{\textcolor[rgb]{0.10,0.09,0.49}{##1}}}
\expandafter\def\csname PY@tok@sa\endcsname{\def\PY@tc##1{\textcolor[rgb]{0.73,0.13,0.13}{##1}}}
\expandafter\def\csname PY@tok@sb\endcsname{\def\PY@tc##1{\textcolor[rgb]{0.73,0.13,0.13}{##1}}}
\expandafter\def\csname PY@tok@sc\endcsname{\def\PY@tc##1{\textcolor[rgb]{0.73,0.13,0.13}{##1}}}
\expandafter\def\csname PY@tok@dl\endcsname{\def\PY@tc##1{\textcolor[rgb]{0.73,0.13,0.13}{##1}}}
\expandafter\def\csname PY@tok@s2\endcsname{\def\PY@tc##1{\textcolor[rgb]{0.73,0.13,0.13}{##1}}}
\expandafter\def\csname PY@tok@sh\endcsname{\def\PY@tc##1{\textcolor[rgb]{0.73,0.13,0.13}{##1}}}
\expandafter\def\csname PY@tok@s1\endcsname{\def\PY@tc##1{\textcolor[rgb]{0.73,0.13,0.13}{##1}}}
\expandafter\def\csname PY@tok@mb\endcsname{\def\PY@tc##1{\textcolor[rgb]{0.40,0.40,0.40}{##1}}}
\expandafter\def\csname PY@tok@mf\endcsname{\def\PY@tc##1{\textcolor[rgb]{0.40,0.40,0.40}{##1}}}
\expandafter\def\csname PY@tok@mh\endcsname{\def\PY@tc##1{\textcolor[rgb]{0.40,0.40,0.40}{##1}}}
\expandafter\def\csname PY@tok@mi\endcsname{\def\PY@tc##1{\textcolor[rgb]{0.40,0.40,0.40}{##1}}}
\expandafter\def\csname PY@tok@il\endcsname{\def\PY@tc##1{\textcolor[rgb]{0.40,0.40,0.40}{##1}}}
\expandafter\def\csname PY@tok@mo\endcsname{\def\PY@tc##1{\textcolor[rgb]{0.40,0.40,0.40}{##1}}}
\expandafter\def\csname PY@tok@ch\endcsname{\let\PY@it=\textit\def\PY@tc##1{\textcolor[rgb]{0.25,0.50,0.50}{##1}}}
\expandafter\def\csname PY@tok@cm\endcsname{\let\PY@it=\textit\def\PY@tc##1{\textcolor[rgb]{0.25,0.50,0.50}{##1}}}
\expandafter\def\csname PY@tok@cpf\endcsname{\let\PY@it=\textit\def\PY@tc##1{\textcolor[rgb]{0.25,0.50,0.50}{##1}}}
\expandafter\def\csname PY@tok@c1\endcsname{\let\PY@it=\textit\def\PY@tc##1{\textcolor[rgb]{0.25,0.50,0.50}{##1}}}
\expandafter\def\csname PY@tok@cs\endcsname{\let\PY@it=\textit\def\PY@tc##1{\textcolor[rgb]{0.25,0.50,0.50}{##1}}}

\def\PYZbs{\char`\\}
\def\PYZus{\char`\_}
\def\PYZob{\char`\{}
\def\PYZcb{\char`\}}
\def\PYZca{\char`\^}
\def\PYZam{\char`\&}
\def\PYZlt{\char`\<}
\def\PYZgt{\char`\>}
\def\PYZsh{\char`\#}
\def\PYZpc{\char`\%}
\def\PYZdl{\char`\$}
\def\PYZhy{\char`\-}
\def\PYZsq{\char`\'}
\def\PYZdq{\char`\"}
\def\PYZti{\char`\~}
% for compatibility with earlier versions
\def\PYZat{@}
\def\PYZlb{[}
\def\PYZrb{]}
\makeatother


    % Exact colors from NB
    \definecolor{incolor}{rgb}{0.0, 0.0, 0.5}
    \definecolor{outcolor}{rgb}{0.545, 0.0, 0.0}



    
    % Prevent overflowing lines due to hard-to-break entities
    \sloppy 
    % Setup hyperref package
    \hypersetup{
      breaklinks=true,  % so long urls are correctly broken across lines
      colorlinks=true,
      urlcolor=urlcolor,
      linkcolor=linkcolor,
      citecolor=citecolor,
      }
    % Slightly bigger margins than the latex defaults
    
    \geometry{verbose,tmargin=1in,bmargin=1in,lmargin=1in,rmargin=1in}
    
    

    \begin{document}
    
    
    \maketitle
    
    

    
    \subsection{1. Meet Dr. Ignaz
Semmelweis}\label{meet-dr.-ignaz-semmelweis}

This is Dr. Ignaz Semmelweis, a Hungarian physician born in 1818 and
active at the Vienna General Hospital. If Dr. Semmelweis looks troubled
it's probably because he's thinking about childbed fever: A deadly
disease affecting women that just have given birth. He is thinking about
it because in the early 1840s at the Vienna General Hospital as many as
10\% of the women giving birth die from it. He is thinking about it
because he knows the cause of childbed fever: It's the contaminated
hands of the doctors delivering the babies. And they won't listen to him
and wash their hands!

In this notebook, we're going to reanalyze the data that made Semmelweis
discover the importance of handwashing. Let's start by looking at the
data that made Semmelweis realize that something was wrong with the
procedures at Vienna General Hospital.

    \begin{Verbatim}[commandchars=\\\{\}]
{\color{incolor}In [{\color{incolor}204}]:} \PY{c+c1}{\PYZsh{} importing modules}
          \PY{c+c1}{\PYZsh{} ... YOUR CODE FOR TASK 1 ...}
          \PY{k+kn}{import} \PY{n+nn}{pandas} \PY{k}{as} \PY{n+nn}{pd}
          \PY{k+kn}{import} \PY{n+nn}{matplotlib}\PY{n+nn}{.}\PY{n+nn}{pyplot} \PY{k}{as} \PY{n+nn}{plt}
          \PY{c+c1}{\PYZsh{} Read datasets/yearly\PYZus{}deaths\PYZus{}by\PYZus{}clinic.csv into yearly}
          \PY{n}{yearly} \PY{o}{=} \PY{n}{pd}\PY{o}{.}\PY{n}{read\PYZus{}csv}\PY{p}{(}\PY{l+s+s1}{\PYZsq{}}\PY{l+s+s1}{datasets/yearly\PYZus{}deaths\PYZus{}by\PYZus{}clinic.csv}\PY{l+s+s1}{\PYZsq{}}\PY{p}{)}
          
          \PY{c+c1}{\PYZsh{} Print out yearly}
          \PY{c+c1}{\PYZsh{} ... YOUR CODE FOR TASK 1 ...}
          \PY{n}{yearly}\PY{o}{.}\PY{n}{head}\PY{p}{(}\PY{p}{)}
\end{Verbatim}


\begin{Verbatim}[commandchars=\\\{\}]
{\color{outcolor}Out[{\color{outcolor}204}]:}    year  births  deaths    clinic
          0  1841    3036     237  clinic 1
          1  1842    3287     518  clinic 1
          2  1843    3060     274  clinic 1
          3  1844    3157     260  clinic 1
          4  1845    3492     241  clinic 1
\end{Verbatim}
            
    \subsection{2. The alarming number of
deaths}\label{the-alarming-number-of-deaths}

The table above shows the number of women giving birth at the two
clinics at the Vienna General Hospital for the years 1841 to 1846.
You'll notice that giving birth was very dangerous; an alarming number
of women died as the result of childbirth, most of them from childbed
fever.

We see this more clearly if we look at the proportion of deaths out of
the number of women giving birth. Let's zoom in on the proportion of
deaths at Clinic 1.

    \begin{Verbatim}[commandchars=\\\{\}]
{\color{incolor}In [{\color{incolor}206}]:} \PY{c+c1}{\PYZsh{} Calculate proportion of deaths per no. births}
          \PY{c+c1}{\PYZsh{} ... YOUR CODE FOR TASK 2 ...}
          \PY{n}{yearly}\PY{p}{[}\PY{l+s+s2}{\PYZdq{}}\PY{l+s+s2}{proportion\PYZus{}deaths}\PY{l+s+s2}{\PYZdq{}}\PY{p}{]} \PY{o}{=}  \PY{n}{yearly}\PY{o}{.}\PY{n}{deaths}\PY{o}{.}\PY{n}{divide}\PY{p}{(}\PY{n}{yearly}\PY{o}{.}\PY{n}{births}\PY{p}{,} \PY{n}{axis} \PY{o}{=} \PY{l+s+s1}{\PYZsq{}}\PY{l+s+s1}{rows}\PY{l+s+s1}{\PYZsq{}}\PY{p}{)}
          \PY{c+c1}{\PYZsh{} Extract clinic 1 data into yearly1 and clinic 2 data into yearly2}
          \PY{n}{yearly1} \PY{o}{=} \PY{n}{yearly}\PY{p}{[}\PY{n}{yearly}\PY{p}{[}\PY{l+s+s1}{\PYZsq{}}\PY{l+s+s1}{clinic}\PY{l+s+s1}{\PYZsq{}}\PY{p}{]} \PY{o}{==} \PY{l+s+s1}{\PYZsq{}}\PY{l+s+s1}{clinic 1}\PY{l+s+s1}{\PYZsq{}}\PY{p}{]}
          \PY{n}{yearly2} \PY{o}{=} \PY{n}{yearly}\PY{p}{[}\PY{n}{yearly}\PY{p}{[}\PY{l+s+s1}{\PYZsq{}}\PY{l+s+s1}{clinic}\PY{l+s+s1}{\PYZsq{}}\PY{p}{]} \PY{o}{==} \PY{l+s+s1}{\PYZsq{}}\PY{l+s+s1}{clinic 2}\PY{l+s+s1}{\PYZsq{}}\PY{p}{]}
          
          \PY{c+c1}{\PYZsh{} Print out yearly1}
          \PY{c+c1}{\PYZsh{} ... YOUR CODE FOR TASK 2 ...}
          \PY{n}{yearly1}
\end{Verbatim}


\begin{Verbatim}[commandchars=\\\{\}]
{\color{outcolor}Out[{\color{outcolor}206}]:}    year  births  deaths    clinic  proportion\_deaths
          0  1841    3036     237  clinic 1           0.078063
          1  1842    3287     518  clinic 1           0.157591
          2  1843    3060     274  clinic 1           0.089542
          3  1844    3157     260  clinic 1           0.082357
          4  1845    3492     241  clinic 1           0.069015
          5  1846    4010     459  clinic 1           0.114464
\end{Verbatim}
            
    \subsection{3. Death at the clinics}\label{death-at-the-clinics}

If we now plot the proportion of deaths at both clinic 1 and clinic 2
we'll see a curious pattern...

    \begin{Verbatim}[commandchars=\\\{\}]
{\color{incolor}In [{\color{incolor}208}]:} \PY{c+c1}{\PYZsh{} This makes plots appear in the notebook}
          \PY{o}{\PYZpc{}}\PY{k}{matplotlib} inline
          
          \PY{c+c1}{\PYZsh{} Plot yearly proportion of deaths at the two clinics}
          \PY{c+c1}{\PYZsh{} ... YOUR CODE FOR TASK 3 ...}
          \PY{n}{yearly1}\PY{p}{[}\PY{l+s+s1}{\PYZsq{}}\PY{l+s+s1}{proportion\PYZus{}deaths}\PY{l+s+s1}{\PYZsq{}}\PY{p}{]} \PY{o}{=} \PY{n}{yearly1}\PY{o}{.}\PY{n}{deaths}\PY{o}{.}\PY{n}{divide}\PY{p}{(}\PY{n}{yearly1}\PY{o}{.}\PY{n}{births}\PY{p}{,} \PY{n}{axis} \PY{o}{=} \PY{l+s+s1}{\PYZsq{}}\PY{l+s+s1}{rows}\PY{l+s+s1}{\PYZsq{}}\PY{p}{)}
          \PY{n}{yearly2}\PY{p}{[}\PY{l+s+s1}{\PYZsq{}}\PY{l+s+s1}{proportion\PYZus{}deaths}\PY{l+s+s1}{\PYZsq{}}\PY{p}{]} \PY{o}{=} \PY{n}{yearly2}\PY{o}{.}\PY{n}{deaths}\PY{o}{.}\PY{n}{divide}\PY{p}{(}\PY{n}{yearly2}\PY{o}{.}\PY{n}{births}\PY{p}{,} \PY{n}{axis} \PY{o}{=} \PY{l+s+s1}{\PYZsq{}}\PY{l+s+s1}{rows}\PY{l+s+s1}{\PYZsq{}}\PY{p}{)}
          
          \PY{n}{ax} \PY{o}{=} \PY{n}{yearly1}\PY{o}{.}\PY{n}{plot}\PY{p}{(}\PY{n}{x}\PY{o}{=}\PY{l+s+s2}{\PYZdq{}}\PY{l+s+s2}{year}\PY{l+s+s2}{\PYZdq{}}\PY{p}{,} \PY{n}{y}\PY{o}{=}\PY{l+s+s2}{\PYZdq{}}\PY{l+s+s2}{proportion\PYZus{}deaths}\PY{l+s+s2}{\PYZdq{}}\PY{p}{,} \PY{n}{label}\PY{o}{=}\PY{l+s+s2}{\PYZdq{}}\PY{l+s+s2}{Clinic 1}\PY{l+s+s2}{\PYZdq{}}\PY{p}{)}
          \PY{n}{yearly2}\PY{o}{.}\PY{n}{plot}\PY{p}{(}\PY{n}{x}\PY{o}{=}\PY{l+s+s2}{\PYZdq{}}\PY{l+s+s2}{year}\PY{l+s+s2}{\PYZdq{}}\PY{p}{,} \PY{n}{y}\PY{o}{=}\PY{l+s+s2}{\PYZdq{}}\PY{l+s+s2}{proportion\PYZus{}deaths}\PY{l+s+s2}{\PYZdq{}}\PY{p}{,} \PY{n}{label}\PY{o}{=}\PY{l+s+s2}{\PYZdq{}}\PY{l+s+s2}{Clinic 2}\PY{l+s+s2}{\PYZdq{}}\PY{p}{,} \PY{n}{c} \PY{o}{=} \PY{l+s+s1}{\PYZsq{}}\PY{l+s+s1}{r}\PY{l+s+s1}{\PYZsq{}}\PY{p}{,} \PY{n}{ax} \PY{o}{=} \PY{n}{ax}\PY{p}{)}
          \PY{n}{ax}\PY{o}{.}\PY{n}{set\PYZus{}ylabel}\PY{p}{(}\PY{l+s+s1}{\PYZsq{}}\PY{l+s+s1}{Proportion od death}\PY{l+s+s1}{\PYZsq{}}\PY{p}{)}
\end{Verbatim}


\begin{Verbatim}[commandchars=\\\{\}]
{\color{outcolor}Out[{\color{outcolor}208}]:} <matplotlib.text.Text at 0x7fd59de8ae48>
\end{Verbatim}
            
    \begin{center}
    \adjustimage{max size={0.9\linewidth}{0.9\paperheight}}{output_5_1.png}
    \end{center}
    { \hspace*{\fill} \\}
    
    \subsection{4. The handwashing begins}\label{the-handwashing-begins}

Why is the proportion of deaths constantly so much higher in Clinic 1?
Semmelweis saw the same pattern and was puzzled and distressed. The only
difference between the clinics was that many medical students served at
Clinic 1, while mostly midwife students served at Clinic 2. While the
midwives only tended to the women giving birth, the medical students
also spent time in the autopsy rooms examining corpses.

Semmelweis started to suspect that something on the corpses, spread from
the hands of the medical students, caused childbed fever. So in a
desperate attempt to stop the high mortality rates, he decreed: Wash
your hands! This was an unorthodox and controversial request, nobody in
Vienna knew about bacteria at this point in time.

Let's load in monthly data from Clinic 1 to see if the handwashing had
any effect.

    \begin{Verbatim}[commandchars=\\\{\}]
{\color{incolor}In [{\color{incolor}210}]:} \PY{c+c1}{\PYZsh{} Read datasets/monthly\PYZus{}deaths.csv into monthly}
          \PY{n}{monthly} \PY{o}{=} \PY{n}{pd}\PY{o}{.}\PY{n}{read\PYZus{}csv}\PY{p}{(}\PY{l+s+s1}{\PYZsq{}}\PY{l+s+s1}{datasets/monthly\PYZus{}deaths.csv}\PY{l+s+s1}{\PYZsq{}}\PY{p}{,} \PY{n}{parse\PYZus{}dates} \PY{o}{=} \PY{p}{[}\PY{l+s+s1}{\PYZsq{}}\PY{l+s+s1}{date}\PY{l+s+s1}{\PYZsq{}}\PY{p}{]}\PY{p}{)}
          \PY{c+c1}{\PYZsh{} Calculate proportion of deaths per no. births}
          \PY{c+c1}{\PYZsh{} ... YOUR CODE FOR TASK 4 ...}
          \PY{n}{monthly}\PY{p}{[}\PY{l+s+s1}{\PYZsq{}}\PY{l+s+s1}{proportion\PYZus{}deaths}\PY{l+s+s1}{\PYZsq{}}\PY{p}{]} \PY{o}{=} \PY{n}{monthly}\PY{o}{.}\PY{n}{deaths}\PY{o}{.}\PY{n}{divide}\PY{p}{(}\PY{n}{monthly}\PY{o}{.}\PY{n}{births}\PY{p}{,} \PY{n}{axis} \PY{o}{=} \PY{l+s+s1}{\PYZsq{}}\PY{l+s+s1}{rows}\PY{l+s+s1}{\PYZsq{}}\PY{p}{)}
          \PY{c+c1}{\PYZsh{} Print out the first rows in monthly}
          \PY{c+c1}{\PYZsh{} ... YOUR CODE FOR TASK 4 ...}
          \PY{n}{monthly}\PY{o}{.}\PY{n}{head}\PY{p}{(}\PY{p}{)}
\end{Verbatim}


\begin{Verbatim}[commandchars=\\\{\}]
{\color{outcolor}Out[{\color{outcolor}210}]:}         date  births  deaths  proportion\_deaths
          0 1841-01-01     254      37           0.145669
          1 1841-02-01     239      18           0.075314
          2 1841-03-01     277      12           0.043321
          3 1841-04-01     255       4           0.015686
          4 1841-05-01     255       2           0.007843
\end{Verbatim}
            
    \subsection{5. The effect of
handwashing}\label{the-effect-of-handwashing}

With the data loaded we can now look at the proportion of deaths over
time. In the plot below we haven't marked where obligatory handwashing
started, but it reduced the proportion of deaths to such a degree that
you should be able to spot it!

    \begin{Verbatim}[commandchars=\\\{\}]
{\color{incolor}In [{\color{incolor}212}]:} \PY{c+c1}{\PYZsh{} Plot monthly proportion of deaths}
          \PY{c+c1}{\PYZsh{} ... YOUR CODE FOR TASK 5 ...}
          \PY{n}{ax} \PY{o}{=} \PY{n}{monthly}\PY{o}{.}\PY{n}{plot}\PY{p}{(}\PY{n}{x} \PY{o}{=}\PY{l+s+s1}{\PYZsq{}}\PY{l+s+s1}{date}\PY{l+s+s1}{\PYZsq{}}\PY{p}{,} \PY{n}{y}\PY{o}{=} \PY{l+s+s1}{\PYZsq{}}\PY{l+s+s1}{proportion\PYZus{}deaths}\PY{l+s+s1}{\PYZsq{}}\PY{p}{)}
          \PY{n}{ax}\PY{o}{.}\PY{n}{set\PYZus{}ylabel}\PY{p}{(}\PY{l+s+s1}{\PYZsq{}}\PY{l+s+s1}{proportion of deaths}\PY{l+s+s1}{\PYZsq{}}\PY{p}{)}
\end{Verbatim}


\begin{Verbatim}[commandchars=\\\{\}]
{\color{outcolor}Out[{\color{outcolor}212}]:} <matplotlib.text.Text at 0x7fd59d8b2278>
\end{Verbatim}
            
    \begin{center}
    \adjustimage{max size={0.9\linewidth}{0.9\paperheight}}{output_9_1.png}
    \end{center}
    { \hspace*{\fill} \\}
    
    \subsection{6. The effect of handwashing
highlighted}\label{the-effect-of-handwashing-highlighted}

Starting from the summer of 1847 the proportion of deaths is drastically
reduced and, yes, this was when Semmelweis made handwashing obligatory.

The effect of handwashing is made even more clear if we highlight this
in the graph.

    \begin{Verbatim}[commandchars=\\\{\}]
{\color{incolor}In [{\color{incolor}214}]:} \PY{c+c1}{\PYZsh{} Date when handwashing was made mandatory}
          \PY{k+kn}{import} \PY{n+nn}{pandas} \PY{k}{as} \PY{n+nn}{pd}
          \PY{n}{handwashing\PYZus{}start} \PY{o}{=} \PY{n}{pd}\PY{o}{.}\PY{n}{to\PYZus{}datetime}\PY{p}{(}\PY{l+s+s1}{\PYZsq{}}\PY{l+s+s1}{1847\PYZhy{}06\PYZhy{}01}\PY{l+s+s1}{\PYZsq{}}\PY{p}{)}
          
          \PY{c+c1}{\PYZsh{} Split monthly into before and after handwashing\PYZus{}start}
          \PY{n}{before\PYZus{}washing} \PY{o}{=} \PY{n}{monthly}\PY{p}{[}\PY{n}{monthly}\PY{p}{[}\PY{l+s+s2}{\PYZdq{}}\PY{l+s+s2}{date}\PY{l+s+s2}{\PYZdq{}}\PY{p}{]} \PY{o}{\PYZlt{}} \PY{n}{handwashing\PYZus{}start}\PY{p}{]}
          \PY{n}{after\PYZus{}washing} \PY{o}{=} \PY{n}{monthly}\PY{p}{[}\PY{n}{monthly}\PY{p}{[}\PY{l+s+s2}{\PYZdq{}}\PY{l+s+s2}{date}\PY{l+s+s2}{\PYZdq{}}\PY{p}{]} \PY{o}{\PYZgt{}}\PY{o}{=} \PY{n}{handwashing\PYZus{}start}\PY{p}{]}
          
          \PY{c+c1}{\PYZsh{} Plot monthly proportion of deaths before and after handwashing}
          \PY{c+c1}{\PYZsh{} ... YOUR CODE FOR TASK 6 ...}
          \PY{n}{ax} \PY{o}{=} \PY{n}{before\PYZus{}washing}\PY{o}{.}\PY{n}{plot}\PY{p}{(}\PY{n}{x}\PY{o}{=} \PY{l+s+s1}{\PYZsq{}}\PY{l+s+s1}{date}\PY{l+s+s1}{\PYZsq{}}\PY{p}{,} \PY{n}{y}\PY{o}{=}\PY{l+s+s1}{\PYZsq{}}\PY{l+s+s1}{proportion\PYZus{}deaths}\PY{l+s+s1}{\PYZsq{}}\PY{p}{)}
          \PY{n}{after\PYZus{}washing}\PY{o}{.}\PY{n}{plot}\PY{p}{(}\PY{n}{x}\PY{o}{=} \PY{l+s+s1}{\PYZsq{}}\PY{l+s+s1}{date}\PY{l+s+s1}{\PYZsq{}}\PY{p}{,} \PY{n}{y}\PY{o}{=}\PY{l+s+s1}{\PYZsq{}}\PY{l+s+s1}{proportion\PYZus{}deaths}\PY{l+s+s1}{\PYZsq{}}\PY{p}{,} \PY{n}{ax} \PY{o}{=} \PY{n}{ax}\PY{p}{)}
          \PY{n}{ax}\PY{o}{.}\PY{n}{set\PYZus{}ylabel}\PY{p}{(}\PY{l+s+s1}{\PYZsq{}}\PY{l+s+s1}{proportion of deaths}\PY{l+s+s1}{\PYZsq{}}\PY{p}{)}
\end{Verbatim}


\begin{Verbatim}[commandchars=\\\{\}]
{\color{outcolor}Out[{\color{outcolor}214}]:} <matplotlib.text.Text at 0x7fd59ffdb4e0>
\end{Verbatim}
            
    \begin{center}
    \adjustimage{max size={0.9\linewidth}{0.9\paperheight}}{output_11_1.png}
    \end{center}
    { \hspace*{\fill} \\}
    
    \subsection{7. More handwashing, fewer
deaths?}\label{more-handwashing-fewer-deaths}

Again, the graph shows that handwashing had a huge effect. How much did
it reduce the monthly proportion of deaths on average?

    \begin{Verbatim}[commandchars=\\\{\}]
{\color{incolor}In [{\color{incolor}216}]:} \PY{c+c1}{\PYZsh{} Difference in mean monthly proportion of deaths due to handwashing}
          \PY{n}{before\PYZus{}proportion} \PY{o}{=} \PY{n}{before\PYZus{}washing}\PY{o}{.}\PY{n}{proportion\PYZus{}deaths}
          \PY{n}{after\PYZus{}proportion} \PY{o}{=} \PY{n}{after\PYZus{}washing}\PY{o}{.}\PY{n}{proportion\PYZus{}deaths}
          \PY{n}{mean\PYZus{}diff} \PY{o}{=} \PY{n}{after\PYZus{}proportion}\PY{o}{.}\PY{n}{mean}\PY{p}{(}\PY{p}{)} \PY{o}{\PYZhy{}} \PY{n}{before\PYZus{}proportion}\PY{o}{.}\PY{n}{mean}\PY{p}{(}\PY{p}{)}
          \PY{n}{mean\PYZus{}diff}
\end{Verbatim}


\begin{Verbatim}[commandchars=\\\{\}]
{\color{outcolor}Out[{\color{outcolor}216}]:} -0.08395660751183336
\end{Verbatim}
            
    \subsection{8. A Bootstrap analysis of Semmelweis handwashing
data}\label{a-bootstrap-analysis-of-semmelweis-handwashing-data}

It reduced the proportion of deaths by around 8 percentage points! From
10\% on average to just 2\% (which is still a high number by modern
standards).

To get a feeling for the uncertainty around how much handwashing reduces
mortalities we could look at a confidence interval (here calculated
using the bootstrap method).

    \begin{Verbatim}[commandchars=\\\{\}]
{\color{incolor}In [{\color{incolor}218}]:} \PY{c+c1}{\PYZsh{} A bootstrap analysis of the reduction of deaths due to handwashing}
          \PY{n}{boot\PYZus{}mean\PYZus{}diff} \PY{o}{=} \PY{p}{[}\PY{p}{]}
          \PY{k}{for} \PY{n}{i} \PY{o+ow}{in} \PY{n+nb}{range}\PY{p}{(}\PY{l+m+mi}{3000}\PY{p}{)}\PY{p}{:}
              \PY{n}{boot\PYZus{}before} \PY{o}{=} \PY{n}{before\PYZus{}proportion}\PY{o}{.}\PY{n}{sample}\PY{p}{(}\PY{n}{frac}\PY{o}{=}\PY{l+m+mi}{1}\PY{p}{,} \PY{n}{replace}\PY{o}{=}\PY{k+kc}{True}\PY{p}{)}
              \PY{n}{boot\PYZus{}after} \PY{o}{=} \PY{n}{after\PYZus{}proportion}\PY{o}{.}\PY{n}{sample}\PY{p}{(}\PY{n}{frac}\PY{o}{=}\PY{l+m+mi}{1}\PY{p}{,} \PY{n}{replace}\PY{o}{=}\PY{k+kc}{True}\PY{p}{)}
              \PY{n}{boot\PYZus{}mean\PYZus{}diff}\PY{o}{.}\PY{n}{append}\PY{p}{(}\PY{n}{boot\PYZus{}after}\PY{o}{.}\PY{n}{mean}\PY{p}{(}\PY{p}{)} \PY{o}{\PYZhy{}} \PY{n}{boot\PYZus{}before}\PY{o}{.}\PY{n}{mean}\PY{p}{(}\PY{p}{)}\PY{p}{)}
          
          \PY{c+c1}{\PYZsh{} Calculating a 95\PYZpc{} confidence interval from boot\PYZus{}mean\PYZus{}diff }
          \PY{n}{confidence\PYZus{}interval} \PY{o}{=} \PY{n}{pd}\PY{o}{.}\PY{n}{Series}\PY{p}{(}\PY{n}{boot\PYZus{}mean\PYZus{}diff}\PY{p}{)}\PY{o}{.}\PY{n}{quantile}\PY{p}{(}\PY{p}{[}\PY{l+m+mf}{0.025}\PY{p}{,} \PY{l+m+mf}{0.975}\PY{p}{]}\PY{p}{)}
          \PY{n}{confidence\PYZus{}interval}
\end{Verbatim}


\begin{Verbatim}[commandchars=\\\{\}]
{\color{outcolor}Out[{\color{outcolor}218}]:} 0.025   -0.100979
          0.975   -0.067142
          dtype: float64
\end{Verbatim}
            
    \subsection{9. The fate of Dr.
Semmelweis}\label{the-fate-of-dr.-semmelweis}

So handwashing reduced the proportion of deaths by between 6.7 and 10
percentage points, according to a 95\% confidence interval. All in all,
it would seem that Semmelweis had solid evidence that handwashing was a
simple but highly effective procedure that could save many lives.

The tragedy is that, despite the evidence, Semmelweis' theory --- that
childbed fever was caused by some "substance" (what we today know as
bacteria) from autopsy room corpses --- was ridiculed by contemporary
scientists. The medical community largely rejected his discovery and in
1849 he was forced to leave the Vienna General Hospital for good.

One reason for this was that statistics and statistical arguments were
uncommon in medical science in the 1800s. Semmelweis only published his
data as long tables of raw data, but he didn't show any graphs nor
confidence intervals. If he would have had access to the analysis we've
just put together he might have been more successful in getting the
Viennese doctors to wash their hands.

    \begin{Verbatim}[commandchars=\\\{\}]
{\color{incolor}In [{\color{incolor}220}]:} \PY{c+c1}{\PYZsh{} The data Semmelweis collected points to that:}
          \PY{n}{doctors\PYZus{}should\PYZus{}wash\PYZus{}their\PYZus{}hands} \PY{o}{=} \PY{k+kc}{False}
\end{Verbatim}



    % Add a bibliography block to the postdoc
    
    
    
    \end{document}
